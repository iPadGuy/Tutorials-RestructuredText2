%% Generated by Sphinx.
\def\sphinxdocclass{report}
\documentclass[letterpaper,10pt,english]{sphinxmanual}
\ifdefined\pdfpxdimen
   \let\sphinxpxdimen\pdfpxdimen\else\newdimen\sphinxpxdimen
\fi \sphinxpxdimen=.75bp\relax

\PassOptionsToPackage{warn}{textcomp}
\usepackage[utf8]{inputenc}
\ifdefined\DeclareUnicodeCharacter
% support both utf8 and utf8x syntaxes
  \ifdefined\DeclareUnicodeCharacterAsOptional
    \def\sphinxDUC#1{\DeclareUnicodeCharacter{"#1}}
  \else
    \let\sphinxDUC\DeclareUnicodeCharacter
  \fi
  \sphinxDUC{00A0}{\nobreakspace}
  \sphinxDUC{2500}{\sphinxunichar{2500}}
  \sphinxDUC{2502}{\sphinxunichar{2502}}
  \sphinxDUC{2514}{\sphinxunichar{2514}}
  \sphinxDUC{251C}{\sphinxunichar{251C}}
  \sphinxDUC{2572}{\textbackslash}
\fi
\usepackage{cmap}
\usepackage[T1]{fontenc}
\usepackage{amsmath,amssymb,amstext}
\usepackage{babel}



\usepackage{times}
\expandafter\ifx\csname T@LGR\endcsname\relax
\else
% LGR was declared as font encoding
  \substitutefont{LGR}{\rmdefault}{cmr}
  \substitutefont{LGR}{\sfdefault}{cmss}
  \substitutefont{LGR}{\ttdefault}{cmtt}
\fi
\expandafter\ifx\csname T@X2\endcsname\relax
  \expandafter\ifx\csname T@T2A\endcsname\relax
  \else
  % T2A was declared as font encoding
    \substitutefont{T2A}{\rmdefault}{cmr}
    \substitutefont{T2A}{\sfdefault}{cmss}
    \substitutefont{T2A}{\ttdefault}{cmtt}
  \fi
\else
% X2 was declared as font encoding
  \substitutefont{X2}{\rmdefault}{cmr}
  \substitutefont{X2}{\sfdefault}{cmss}
  \substitutefont{X2}{\ttdefault}{cmtt}
\fi


\usepackage[Bjarne]{fncychap}
\usepackage{sphinx}

\fvset{fontsize=\small}
\usepackage{geometry}

% Include hyperref last.
\usepackage{hyperref}
% Fix anchor placement for figures with captions.
\usepackage{hypcap}% it must be loaded after hyperref.
% Set up styles of URL: it should be placed after hyperref.
\urlstyle{same}
\addto\captionsenglish{\renewcommand{\contentsname}{Contents:}}

\usepackage{sphinxmessages}
\setcounter{tocdepth}{1}



\title{Bananas}
\date{Oct 28, 2019}
\release{1.0}
\author{iPad Guy}
\newcommand{\sphinxlogo}{\vbox{}}
\renewcommand{\releasename}{Release}
\makeindex
\begin{document}

\pagestyle{empty}
\sphinxmaketitle
\pagestyle{plain}
\sphinxtableofcontents
\pagestyle{normal}
\phantomsection\label{\detokenize{index::doc}}



\chapter{Document Title}
\label{\detokenize{sample:document-title}}\label{\detokenize{sample::doc}}

\section{Subtitle}
\label{\detokenize{sample:subtitle}}
\begin{sphinxShadowBox}
\sphinxstyletopictitle{Overview}
\begin{itemize}
\item {} 
\phantomsection\label{\detokenize{sample:id4}}{\hyperref[\detokenize{sample:document-title}]{\sphinxcrossref{Document Title}}}
\begin{itemize}
\item {} 
\phantomsection\label{\detokenize{sample:id5}}{\hyperref[\detokenize{sample:subtitle}]{\sphinxcrossref{Subtitle}}}
\begin{itemize}
\item {} 
\phantomsection\label{\detokenize{sample:id6}}{\hyperref[\detokenize{sample:section-1}]{\sphinxcrossref{Section 1}}}

\item {} 
\phantomsection\label{\detokenize{sample:id7}}{\hyperref[\detokenize{sample:examples}]{\sphinxcrossref{Examples}}}

\end{itemize}

\end{itemize}

\end{itemize}
\end{sphinxShadowBox}


\subsection{Section 1}
\label{\detokenize{sample:section-1}}
Text can be \sphinxstyleemphasis{italicized} or \sphinxstylestrong{bolded}  as well as \sphinxcode{\sphinxupquote{monospaced}}.
You can *escape certain* special characters.


\subsubsection{Subsection 1 (Level 2)}
\label{\detokenize{sample:subsection-1-level-2}}
Some section 2 text


\paragraph{Sub-subsection 1 (level 3)}
\label{\detokenize{sample:sub-subsection-1-level-3}}
Some more text.


\subsection{Examples}
\label{\detokenize{sample:examples}}

\subsubsection{Comments}
\label{\detokenize{sample:comments}}
Special notes that are not shown but might come out as HTML comments


\subsubsection{Images}
\label{\detokenize{sample:images}}
Add an image with:

\noindent\sphinxincludegraphics[width=200\sphinxpxdimen,height=100\sphinxpxdimen]{{screenshots/file}.png}

You can inline an image or other directive with the \sphinxincludegraphics{{image/image}.png} command.


\subsubsection{Lists}
\label{\detokenize{sample:lists}}\begin{itemize}
\item {} 
Bullet are made like this

\item {} \begin{description}
\item[{Point levels must be consistent}] \leavevmode\begin{itemize}
\item {} \begin{description}
\item[{Sub-bullets}] \leavevmode\begin{itemize}
\item {} 
Sub-sub-bullets

\end{itemize}

\end{description}

\end{itemize}

\end{description}

\item {} 
Lists

\end{itemize}
\begin{description}
\item[{Term}] \leavevmode
Definition for term

\item[{Term2}] \leavevmode
Definition for term 2

\end{description}
\begin{quote}\begin{description}
\item[{List of Things}] \leavevmode
item1 - these are ‘field lists’ not bulleted lists
item2
item 3

\item[{Something}] \leavevmode
single item

\item[{Someitem}] \leavevmode
single item

\end{description}\end{quote}


\subsubsection{Preformatted text}
\label{\detokenize{sample:preformatted-text}}
A code example prefix must always end with double colon like it’s presenting something:

\begin{sphinxVerbatim}[commandchars=\\\{\}]
   \PYG{n}{Anything} \PYG{n}{indented} \PYG{o+ow}{is} \PYG{n}{part} \PYG{n}{of} \PYG{n}{the} \PYG{n}{preformatted} \PYG{n}{block}
  \PYG{n}{Until}
 \PYG{n}{It} \PYG{n}{gets} \PYG{n}{back} \PYG{n}{to}
\PYG{n}{Allll} \PYG{n}{the} \PYG{n}{way} \PYG{n}{left}
\end{sphinxVerbatim}

Now we’re out of the preformatted block.


\subsubsection{Code blocks}
\label{\detokenize{sample:code-blocks}}
There are three equivalents: \sphinxcode{\sphinxupquote{code}}, \sphinxcode{\sphinxupquote{sourcecode}}, and \sphinxcode{\sphinxupquote{code-block}}.

\begin{sphinxVerbatim}[commandchars=\\\{\}]
\PYG{k+kn}{import} \PYG{n+nn}{os}
\PYG{k}{print}\PYG{p}{(}\PYG{n}{help}\PYG{p}{(}\PYG{n}{os}\PYG{p}{)}\PYG{p}{)}
\end{sphinxVerbatim}

\begin{sphinxVerbatim}[commandchars=\\\{\}]
\PYG{c+c1}{\PYGZsh{} Equivalent}
\end{sphinxVerbatim}

\begin{sphinxVerbatim}[commandchars=\\\{\}]
\PYG{c+c1}{\PYGZsh{} Equivalent}
\end{sphinxVerbatim}


\subsubsection{Links}
\label{\detokenize{sample:links}}
Web addresses by themselves will auto link, like this: \sphinxurl{https://www.devdungeon.com}

You can also inline custom links: \sphinxhref{https://www.google.com}{Google search engine}

This is a simple \sphinxhref{https://www.google.com}{link} to Google with the link defined separately.

This is a link to the \sphinxhref{http://www.python.org/}{Python website}.

This is a link back to {\hyperref[\detokenize{sample:section-1}]{\sphinxcrossref{Section 1}}}. You can link based off of the heading name
within a document.


\subsubsection{Footnotes}
\label{\detokenize{sample:footnotes}}
Footnote Reference %
\begin{footnote}[1]\sphinxAtStartFootnote
This is footnote number one that would go at the bottom of the document.
%
\end{footnote}

Or autonumbered {[}\#{]}


\subsubsection{Lines/Transitions}
\label{\detokenize{sample:lines-transitions}}
Any 4+ repeated characters with blank lines surrounding it becomes an hr line, like this.


\bigskip\hrule\bigskip



\subsubsection{Tables}
\label{\detokenize{sample:tables}}

\begin{savenotes}\sphinxattablestart
\centering
\begin{tabulary}{\linewidth}[t]{|T|T|T|}
\hline
\sphinxstyletheadfamily 
Time
&\sphinxstyletheadfamily 
Number
&\sphinxstyletheadfamily 
Value
\\
\hline
12:00
&
42
&
2
\\
\hline
23:00
&
23
&
4
\\
\hline
\end{tabulary}
\par
\sphinxattableend\end{savenotes}


\subsubsection{Preserving line breaks}
\label{\detokenize{sample:preserving-line-breaks}}
Normally you can break the line in the middle of a paragraph and it will
ignore the newline. If you want to preserve the newlines, use the \sphinxcode{\sphinxupquote{\textbar{}}} prefix
on the lines. For example:

\begin{DUlineblock}{0em}
\item[] These lines will
\item[] break exactly
\item[] where we told them to.
\end{DUlineblock}


\chapter{Indices and tables}
\label{\detokenize{index:indices-and-tables}}\begin{itemize}
\item {} 
\DUrole{xref,std,std-ref}{genindex}

\item {} 
\DUrole{xref,std,std-ref}{modindex}

\item {} 
\DUrole{xref,std,std-ref}{search}

\end{itemize}



\renewcommand{\indexname}{Index}
\printindex
\end{document}